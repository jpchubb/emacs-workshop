\documentclass[12pt,a4paper,oneside]{beamer}

%% \usetheme{default} % kind of dull
%% \usetheme{Boadilla} % meh
%% \usetheme{Madrid} % sort of Boadilla / Copenhagen
%% \usetheme{Pittsburgh} % right aligned default
%% \usetheme[height=7mm]{Rochester} % Copenhagen-esque without the bottom bar
%% \usetheme{Copenhagen} % nice
\usetheme{Warsaw} % very nice
%% \usetheme{Singapore} % relaxed
%% \usetheme{Malmoe} % default / Copenhagen

\usepackage{url}
\usepackage{listings}

\title{Introduction to GNU Emacs 23.1.1}
\subtitle{An Advanced Lisp Environment}
\author{Luke English, Joshua Chubb and Matthew Ball}
\date{}

\begin{document}

\begin{frame}
\titlepage
\end{frame}

\begin{frame}{Overview}
This is a tutorial presentation on using GNU Emacs. Included with this presentation is a short article which expands on the concepts introduced here.\newline

\begin{itemize}
  \item Download GNU Emacs: \url{http://www.gnu.org/software/emacs/}\newline
  \item Information and Community: \url{http://www.emacswiki.org/}
\end{itemize}

\end{frame}

\begin{frame}{Introduction}
\begin{itemize}
  \item ``The extensible, customizable, self-documenting, real-time display editor''\newline
  \item Long and extensive history - starting in the 70's and continuing to this present day\newline
  \item First and foremost an \emph{interactive toplevel} programming environment for \texttt{lisp}\newline
  \item This environment provides the user with real-time control of the application; a constant cycle of \texttt{read}, \texttt{evaluate}, \texttt{print} and \texttt{loop}
\end{itemize}
\end{frame}

 %% code \emph{is} data and data \emph{is} code

\begin{frame}{Features}
\begin{itemize}
  \item Commands to manipulate words and paragraphs (deleting them, moving them, moving through them, and so forth)\newline
  \item Intelligent syntax highlighting for making source code easier to read\newline
  \item ``Keyboard macros'' for performing arbitrary batches of editing commands defined by the user\newline
  \item Completely customisable advanced editing environment
\end{itemize}
\end{frame}

\begin{frame}{Terminology}
\begin{itemize}
  \item The single character \texttt{C} represents the \emph{control} key (usually denoted by \texttt{ctrl})\newline
  \item The single character \texttt{M} represents the \emph{meta} key (usually denoted by \texttt{alt})\footnote{The key \texttt{ESC} was originally the \emph{meta} key, and can still be used for such a purpose, however it is easier for a user to reach the standard \texttt{alt} key}\newline
  \item The single character \texttt{S} represents the \emph{shift} key (usually denoted by \texttt{Shift})
\end{itemize}
\end{frame}

\begin{frame}{Terminology}
\begin{itemize}
  \item \textbf{Buffers} - GNU Emacs loads files into internal buffers; it \emph{loads} files, but \emph{saves} buffers.\newline
  \item \textbf{Frames} - One GNU Emacs process can produce a number of \texttt{X} windows. In GNU Emacs' terminology, each of these windows is called a frame.\newline
  \item \textbf{Windows} - Each frame can be split into multiple sections which GNU Emacs calls windows.\newline
  \item \textbf{Point} - In GNU Emacs, the point is the current location of the cursor.
\end{itemize}
\end{frame}

\begin{frame}{Commands and Functions}
\begin{itemize}
  \item Everything is a \emph{command}\newline
  \item Interactive commands are \emph{functions}\newline
  \item Global actions: \texttt{C-x} prefix\newline
  \item Local actions: \texttt{C-c} prefix\newline
  \item Command menu: \texttt{M-x} (all interactive commands are accessible from here)
\end{itemize}
\end{frame}

\begin{frame}{Commands and Functions}
\begin{itemize}
  \item GNU Emacs behaves like other text editors: the character keys (\texttt{a}, \texttt{b}, \texttt{c}, \texttt{1}, \texttt{2}, \texttt{3}, etc) insert characters, the arrow keys move the cursor, backspace deletes text, and so forth\newline
  \item Users invoke other commands with modified keystrokes (called ``chords''): pressing the \texttt{ctrl} key (or \texttt{alt}, \texttt{ESC} keys) in conjunction with a regular key\newline
  \item Some commands work by calling an external program, parsing the program's output, and displaying the results in a GNU Emacs buffer
\end{itemize}
\end{frame}

\begin{frame}{Navigation}
\begin{itemize}
  \item Move forward a character: \texttt{C-f}\newline
  \item Move backwards a character: \texttt{C-b}\newline
  \item Move forward a word: \texttt{M-f}\newline
  \item Move backwards a word \texttt{M-b}\newline
  \item Move to the beginning of a line: \texttt{C-a}
\end{itemize}
\end{frame}

\begin{frame}{Navigation}
\begin{itemize}
  \item Move to the end of a line: \texttt{C-e}\newline
  \item Move to the beginning of a sentence: \texttt{M-a}\newline
  \item Move to the end of a sentence: \texttt{M-e}\newline
  \item Move to the beginning of a buffer: \texttt{M-<}\newline
  \item Move to the end of a buffer: \texttt{M->}
\end{itemize} 
\end{frame}

\begin{frame}{Navigation}
\begin{itemize}
  \item Move down a line: \texttt{C-n}\newline
  \item Move up a line: \texttt{C-p}\newline
  \item Move down half a page: \texttt{C-v}\newline
  \item Move up half a page: \texttt{M-v}\newline
  \item Set cursor in buffer: \texttt{C-l}
\end{itemize}
\end{frame}

\begin{frame}{Manipulating Text}
\begin{itemize}
  \item Delete the next character: \texttt{C-d} (or \texttt{Del})\newline
  \item Delete the previous character: \texttt{Backspace}\newline
  \item Delete the next word: \texttt{M-d} (or \texttt{M-Backspace})\newline
  \item Delete a whole line (or \emph{yank}\footnote{Deleting anything bigger than a character will copy the text to a buffer (the \emph{kill-ring}), the user can paste from this buffer with \texttt{C-y} or cycle through the buffer with \texttt{M-y}.}): \texttt{C-k}\newline
  \item Delete a whole sentence: \texttt{M-k}
\end{itemize}
\end{frame}

\begin{frame}{Search and Replace}
\begin{itemize}
  \item Incremental search: \texttt{C-s}\newline
  \item Decremental search: \texttt{C-r}\newline
  \item Most recently searched item: \texttt{C-s C-s}\newline
  \item Previous item in search history: \texttt{C-s M-p}\newline
  \item Next item in search history: \texttt{C-s M-n}
\end{itemize}
\end{frame}

\begin{frame}{File Handling}
\begin{itemize}
  \item Find a file: \texttt{C-x C-f}\newline
  \item Insert file at cursor: \texttt{C-x i}\newline
  \item Save current buffer: \texttt{C-x C-s}\newline
  \item Write current file: \texttt{C-x C-w}
\end{itemize}
\end{frame}

\begin{frame}{Modes}
\begin{itemize}
  \item Two kinds of modes in GNU Emacs: \emph{Major} modes, and \emph{Minor} modes\newline
  \item Major modes change the \emph{function} of GNU Emacs; for instance the mode \texttt{dired} provides a directory editor in GNU Emacs\newline
  \item Minor modes \emph{enhance} major modes; for instance \texttt{flyspell-mode} provides a dynamic spell-check\newline
  \item Modes are what GNU Emacs equally useful for writing documentation, programming in a variety of languages, sending e-mail, reading usenet postings, tracking appointments, and even playing games\newline
\end{itemize}
\end{frame}

\begin{frame}{Major Modes}
\begin{itemize}
  \item Major modes fall into three groups:\newline
  \item The first group contains modes for normal text; it includes: \texttt{text-mode}, \texttt{html-mode}, \texttt{latex-mode}, \texttt{outline-mode}, etc\ \newline
  \item The second group contains modes for specific programming languages; it includes: \texttt{lisp-mode}, \texttt{c-mode}, \texttt{haskell-mode}, etc\ \newline
  \item The third group contains modes not intended for use on files; it includes: \texttt{dired-mode}, \texttt{mail-mode}, \texttt{shell-mode}, \texttt{erc-mode}, etc\
\end{itemize}
\end{frame}

\begin{frame}{Major Modes}
\begin{itemize}
  \item Popular major modes include:
\end{itemize}
\begin{multicols}
  \begin{tabular}{@{}ll@{}}
    \texttt{fundamental-mode} & default major mode\\
    \texttt{text-mode} & for writing text documents\\
    \texttt{c-mode} & for programming in C\\
    \texttt{c++-mode} & for programming in C++\\
    \texttt{python-mode} & for programming in Python\\
    \texttt{haskell-mode} & for programming in Haskell\\
    \texttt{lisp-mode} & for programming in Lisp\\
    \texttt{latex-mode} & for writing {\LaTeX} documents\\
    \texttt{org-mode} & for organising tasks
  \end{tabular}
\end{multicols}
\end{frame}

\begin{frame}{Minor Modes}
\begin{itemize}
  \item Minor modes may affect documents directly; for instance, the major mode for the C programming language defines a different minor mode for each of the popular indentation styles.\newline
  \item Minor modes may affect the editing environment; for example, \texttt{winner-mode} adds the ability to undo changes to the window configuration.\newline
  \item Minor modes can give mode-specific behaviour; \texttt{flyspell-mode} on a text buffer checks the spelling of a document as it is being typed. In a source file however, the spell check is only applied over the comments.
\end{itemize}
\end{frame}

\begin{frame}{Minor Modes}
\begin{itemize}
  \item Popular minor modes include:
\end{itemize}
\begin{multicols}
  \begin{tabular}{@{}ll@{}}
  \texttt{flyspell-mode} & for spell-checking documents\\
  \texttt{flymake-mode} & for compiler-checking of source code\\
  \texttt{auto-complete-mode} & will auto-complete snippets of text\\
  \texttt{ibuffer} & filters and then groups the list of buffers\\
  \texttt{autopair-mode} & for automatically inserting parenthesis\\
  \texttt{hs-minor-mode} & for code folding\\
  \texttt{reftex-mode} & for preparing {\LaTeX} references
  \end{tabular}
\end{multicols}
\end{frame}

\begin{frame}{Extensions}
\begin{itemize}
  \item Extensions are \emph{global} minor modes which change (or enhance) GNU Emacs' feature-set.\newline
  \item Popular extensions include:
\end{itemize}
\begin{multicols}
  \begin{tabular}{@{}ll@{}}
    \texttt{ido-mode} & improves user interaction with the minibuffer\\
    \texttt{smex} & improves user interaction with the minibuffer\\
    \texttt{iswitchb} & improves user interaction with the buffer list\\
    \texttt{ibuffer} & ibuffer is a replacement for the default \texttt{buffermenu}\\
    \texttt{dired} & a file and directory manager for GNU Emacs
  \end{tabular}
\end{multicols}
\end{frame}

\begin{frame}{Help}
\begin{itemize}
  \item Some useful help commands are listed below:
\end{itemize}
\begin{tabular}{@{}ll@{}}
  \texttt{C-h t} & runs the GNU Emacs tutorial\\
  \texttt{C-h i} & runs the GNU Emacs info system\\
  \texttt{C-h f} & describes an Emacs Lisp function\\
  \texttt{C-h v} & describes an Emacs Lisp variable\\
  \texttt{C-h k} & describes a command bound to a key-binding\\
  \texttt{C-h c} & shows what command a key is bound to\\
  \texttt{C-h a} & search command names using a regular expression\\
  \texttt{C-h m} & describes the current buffer's major mode\\
\end{tabular}
\end{frame}

\begin{frame}{Customising}
\begin{itemize}
  \item Configuring the GNU Emacs extension \texttt{ELPA} (the ``Emacs Lisp Package Archive'') is useful for searching for packages\newline
  \item The package \texttt{emacs-goodies.el} provides a number of new (and improved) modes and features\newline
  \item The \texttt{.emacs} file is used to customise GNU Emacs\newline
  \item Powered by the \texttt{Emacs Lisp} programming language (a variant of the popular \texttt{lisp}): \texttt{Elisp} is used to write extensions and to customise GNU Emacs
\end{itemize}
\end{frame}

\end{document}
