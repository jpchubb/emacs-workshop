\documentclass[12pt,a4paper,oneside]{article}

\usepackage{url}
\usepackage{listings}

\title{Introduction to GNU Emacs 23.1.1}
\author{Matthew Ball}
\date{}

\setcounter{secnumdepth}{0}

\begin{document}

\maketitle\newpage
\tableofcontents\newpage

\section{Overview}
This is a short tutorial article on using GNU Emacs with the GNU/Linux distribution, Ubuntu 10.04.\\

For information and downloads for GNU Emacs please see: \url{http://www.gnu.org/software/emacs/}. For information and downloads of community major modes, minor modes and extensions please see: \url{http://www.emacswiki.org/}.

\subsection{Setting Up}
I am writing this tutorial in \texttt{GNU Emacs 23.1.1 (i486-pc-linux-gnu) of 2010-03-30 on rothera, modified by Debian}. I am using the \texttt{emacs23-nox} build inside a \texttt{GNU Screen} session.\\

This tutorial is \emph{not} limited to only people using the \texttt{nox} variant of GNU Emacs.  The only thing this tutorial will neglect to mention is using the mouse for navigation.  All of the commands and features I talk of below are available using the standard \texttt{emacs23} distribution which \emph{does} make use of the mouse.  If you are using the \texttt{GUI} version of GNU Emacs, take this into account.\\

I have installed a number of extensions to GNU Emacs from various places on the internet; the most useful place for finding information related to GNU Emacs is the fantastic: \url{http://www.emacswiki.org/}. Configuring the GNU Emacs extension \texttt{ELPA} (the ``Emacs Lisp Package Archive'') is useful for finding packages related to GNU Emacs too. If you are using GNU Linux: Ubuntu 10.04 and have kept the default repositories, the package \texttt{emacs-goodies.el} provides a number of new (and improved) modes and features, and the package \texttt{haskell-mode} provides a modern, and intelligent programming mode for the Haskell programming language.

\section{Introduction}
GNU Emacs (as described by the official manual) is ``the extensible, customizable, self-documenting, real-time display editor'' which is freely available for \emph{most} modern operating systems and computers. Development of Emacs, primarily by Richard M. Stallman, began at the MIT AI lab in the early 1970's - the original Emacs software was written solely for the PDP-series of computers. These systems had a keyboard layout which was much different to the standard IBM keyboards we use today.  For this reason, the key-bindings which Emacs uses for manipulating and navigating text, and calling the available function-set seems a bit archaic to newcomers. Many users of Emacs cite this as the reason the editor requires such a learning curve.  In reality though, a user need only commit a few of the key-bindings to memory; most features of Emacs are accessible as a user-defined function which the user can \emph{interact} with.\\

In 1984, Stallman began working on GNU Emacs to produce a free software alternative to the ``Gosling Emacs'' - Gosling Emacs was a proprietary version of an Emacs-like editor written for use on Unix systems.\\

GNU Emacs helps the user be productive by providing an integrated environment for many different kinds of tasks:

\begin{enumerate}
  \item All of the basic editing commands (and there are lots of them) are available no matter what the user is trying to do: write code, read a manual, use a shell, or compose an email.
  \item All the tools GNU Emacs provides for opening, saving, searching, and processing text (and more) are available to the user no matter what they're doing.
\end{enumerate}

This uniformity means that working within GNU Emacs is often easier than learning to use a separate program, especially when that program is liable to have its own set of editing capabilities and shortcuts.\\

If GNU Emacs doesn't work the way the user a would like, they can use the Emacs Lisp (Elisp) language to customize GNU Emacs, automate common tasks, or add new features. Emacs Lisp is very easy to get started with and yet remarkably powerful: the user can use it to alter and extend almost any feature of GNU Emacs. The user can make GNU Emacs whatever they want it to be by writing Emacs Lisp code; one testament to this is the fact that all of the features presented in this introduction are written in Emacs Lisp.

\subsection{Features}
GNU Emacs is primarily a text editor, not a word processor; it concentrates on manipulating pieces of text, rather than manipulating the typeface (the ``font'') of the characters or printing documents (though GNU Emacs can do these as well). Because it is primarily a text-editor, it has included key-bindings which make it possible to leave your hands hovering over the ``home'' row; that is the keys \texttt{a s d} and \texttt{f} for the left side, and the keys \texttt{j k l} and \texttt{;} for the right side. Typically, someone who can touch type on a standard \texttt{QWERTY} keyboard will feel most comfortable with their hands hovering on the home row.\\

GNU Emacs provides commands to manipulate words and paragraphs (deleting them, moving them, moving through them, and so forth), syntax highlighting for making source code easier to read, and ``keyboard macros'' for performing arbitrary batches of editing commands defined by the user. GNU Emacs is a ``real-time display'' editor in that edits get displayed on the screen as they occur.

\subsection{Customisability}
Almost all of the functionality in the GNU Emacs editor, ranging from basic editing operations such as the insertion of characters into a document to the configuration of the user interface, comes under the control of a dialect of the Lisp programming language known as Emacs Lisp. This unique and unusual design provides many of the features found in GNU Emacs. In this Lisp environment, variables and even entire functions can be modified without having to recompile or even restart the editor. Users have three primary ways of customizing GNU Emacs:

\begin{enumerate}
  \item the \emph{customize extension}, which allows the user to set common customization variables, such as the colour scheme, using a graphical interface, etc. This is intended for GNU Emacs beginners who do not want to work with Emacs Lisp code.
  \item collecting keystrokes into macros and replaying them to automate complex, repetitive tasks. This is often done on an ad-hoc basis, with each macro discarded after use, although macros can be saved and invoked later.
  \item using Emacs Lisp. Usually, user-supplied Emacs Lisp code is stored in a file called \texttt{.emacs} and loaded when GNU Emacs starts up. The \texttt{.emacs} file is often used to set variables and key bindings different from the default setting, and to define new commands that the user finds convenient. Many advanced users have \texttt{.emacs} files hundreds of lines long, with idiosyncratic customizations that cause GNU Emacs to diverge wildly from the default behavior.
\end{enumerate}

\subsection{Extensibility}
The behavior of GNU Emacs can be modified almost without limit, either directly by the user, or by loading Emacs Lisp code known variously as ``libraries'', ``packages'', or ``extensions''.\\

GNU Emacs contains a large number of Emacs Lisp libraries, and users can find more ``third-party'' libraries on the Internet. GNU Emacs can be used as an Integrated Development Environment (IDE), allowing programmers to edit, compile, and debug their code within a single interface. Other libraries perform more unusual functions. A few examples include:\\

\begin{multicols}
  \begin{tabular}{@{}ll@{}}
    \texttt{AUCTeX} & a suite of extensions for creating {\TeX} and {\LaTeX} documents.\\
    \texttt{calc} & a powerful reverse-polish notation numerical calculator.\\
    \texttt{calendar} & for keeping appointment calendars and diaries.\\
    \texttt{ediff} & for working with diff files interactively.\\
    \texttt{erc} & an IRC client.\\
    \texttt{gnus} & a full-featured newsreader and email client.\\
    \texttt{org} & for keeping notes, maintaining lists and project planning.\\
  \end{tabular}
\end{multicols}

Many third-party libraries exist on the Internet; for example, there is a library called \texttt{wikipedia-mode} for editing Wikipedia articles. There is even a Usenet newsgroup, \texttt{gnu.emacs.sources}, which is used for posting new libraries. Some third-party libraries eventually make their way into GNU Emacs, thus becoming a ``standard'' library.

\subsection{Terminology}
In the following discussion, the single character \texttt{C} represents the \emph{control} key (usually denoted by \texttt{ctrl}).  While the single character \texttt{M} represents the \emph{meta} key (usually denoted by \texttt{alt}) (The key \texttt{ESC} was originally the meta key, and can still be used for such a purpose, however it is easier for a user to reach the standard alt key).\\

Moreover, as previously mentioned, a lot of the terminology used when talking about GNU Emacs might not be entirely clear to someone coming to GNU Emacs from a Microsoft Windows-based operating system and background. For this reason, it is perhaps helpful to explain some of these terms:\\

\begin{itemize}
  \item \textbf{Buffers} - GNU Emacs loads files into internal buffers; it \emph{loads} files, but \emph{saves} buffers.
  \item \textbf{Frames} - One GNU Emacs process can produce a number of \texttt{X} windows. In GNU Emacs' terminology, each of these windows is called a frame.
  \item \textbf{Windows} - Each frame can be split into multiple sections which GNU Emacs calls windows.
  \item \textbf{Point} - In GNU Emacs, the point is the current location of the cursor.
\end{itemize}

\section{Using GNU Emacs}
Anyone who is interested in using GNU Emacs, should take the tutorial which is included with GNU Emacs. It is an interactive hands-on tutorial which will familiarize the user with many things, including:

\begin{enumerate}
  \item Starting and exiting GNU Emacs.
  \item Basic text movement and editing commands.
  \item Opening and saving files.
  \item GNU Emacs concepts: windows, frames, files, and buffers.
  \item Invoking commands with keybindings and with \texttt{M-x}.
\end{enumerate}

To run the tutorial, start GNU Emacs and type \texttt{C-h t}, that is, \texttt{Ctrl-h} followed by \texttt{t}.

\subsection{Key-bindings and Commands}
In GNU Emacs, \emph{everything} is a command (or function). There exists a distinction between commands the user can use, and commands which GNU Emacs uses. Interactive commands which are frequently used (such as inserting a single character; \texttt{self-insert-command}) are assigned a key-combination, while other commands which are accessible to the user can be accessed from the \texttt{M-x} command menu (this menu can be extended with extensions such as \texttt{ido} and \texttt{smex}, which will be discussed in a following section).\\

In the long run, a user need only commit a few important, commonly used key-bindings to memory; all other commands can be accessed through the use of \texttt{M-x} and the minibuffer. The most important combinations are those which begin with the \texttt{C-x} prefix; usually signifying a \emph{global} action which will effect GNU Emacs (such as exiting the editor). Through the \texttt{C-c} prefix; usually signifying a \emph{local} action (often defined by the user). Or through the use of the previously mentioned \texttt{M-x} prefix.\\

In the normal editing mode, GNU Emacs behaves like other text editors: the character keys (a, b, c, 1, 2, 3, etc.) insert the corresponding characters, the arrow keys move the editing point, backspace deletes text, and so forth. Users invoke other commands with modified keystrokes: pressing the control key and/or the meta key/alt key/Escape key in conjunction with a regular key. Every editing command is actually an invocation of a function in the Emacs Lisp environment. Even a command as simple as typing a to insert the character a involves calling a function; in this case, \texttt{self-insert-command}.\\

Alternatively, users preferring IBM Common User Access style keys can use \texttt{cua-mode}. Some GNU Emacs commands work by invoking an external program (such as \texttt{ispell} for spell-checking or \texttt{gcc} for program compilation), parsing the program's output, and displaying the result in GNU Emacs.\\

Note that the commands \texttt{save-buffer} and \texttt{save-buffers-kill-emacs} use multiple modified keystrokes. For example, \texttt{C-x C-c} means: while holding down the control key, press \texttt{x}; then, while holding down the control key, press \texttt{c}.\\

It may be important to note at this point, that some of the key-bindings presented in this tutorial are \emph{different} to the default settings (and I have not made this explicitly known). To make this easier, I have chosen to not only list each key-binding that I have personally assigned to an Emacs Lisp function, but also listed the exact function which is called by the respective key-binding. In this way, the reader of this tutorial is free to customise their own key-bindings; to make GNU Emacs \emph{their} own editor. In the final section of this tutorial, we will discuss extending GNU Emacs through use of the Emacs Lisp programming language, and the respective \emph{Emacs configuration file}, \texttt{.emacs}. This later section of the tutorial will discuss how a user may change their key-bindings (in any case, if a key-binding listed here is different to a default binding, the user may use the \texttt{M-x <function>} call to issue the command).

\subsection{Minibuffer}
GNU Emacs uses the minibuffer (normally the bottommost line) to request information. Text to target in a search, the name of a file to read or save and similar information is entered in the minibuffer. When applicable, command line completion is usually available using the tab and space keys.

\subsection{File Management}
GNU Emacs keeps text in objects called buffers. The user can create new buffers and dismiss unwanted ones, and several buffers can exist at the same time. Most buffers contain text loaded from text files, which the user can edit and save back to disk. Buffers also serve to store temporary text, such as the documentation strings displayed by the help library.\\

In both text-terminal and graphical modes, GNU Emacs can split the editing area into separate sections (referred to since 1975 as ``windows'', which can be confusing on systems that have another concept of ``windows'' as well), so that more than one buffer can be displayed at a time. This has many uses. For example, one section can be used to display the source code of a program, while another displays the results from compiling the program. In graphical environments, GNU Emacs can also launch multiple graphical-environment windows, known as ``frames'' in the context of GNU Emacs.

\subsection{Cursor Movement}
For navigating around, GNU Emacs boasts a number of incredibly powerful features; to move forward a single character one simple types the key combination \texttt{C-f}, and to move backwards one character one simply types \texttt{C-b}.  To move forward one \emph{word} one simply types \texttt{M-f} and to move backwards one word one simple types \texttt{M-b}.  To move to the beginning of a line one types \texttt{C-a}, and to move the end of a line one types \texttt{C-e}.  To move to the beginning of a sentence one types \texttt{M-a}, and to move to the end of a sentence one types \texttt{M-e}. To move to the beginning of a buffer one types \texttt{M-<}, and to move to the end of a buffer one types \texttt{M->}.  These are among the most commonly used GNU Emacs key-bindings.\\

To scrolle down a single line one uses the \texttt{C-n} key-binding.  To scroll up a single line one uses the \texttt{C-p} key-binding.  To scroll down half a page of lines one uses the \texttt{C-v} command.  To scroll up half a page of lines one uses the \texttt{M-v} command.  To set the screen with the cursor in the middle one types \texttt{C-l}.

\subsection{Manipulating Text}
To delete the next single character one types \texttt{C-d} or \texttt{Del}, to delete the previous single character one types \texttt{Backspace}, to delete the next word one types \texttt{M-d} or \texttt{M-Backspace}.

\subsubsection{Cutting Text}
As with text movement, GNU Emacs provides commands for deleting text in various amounts. To delete a whole line (or in GNU Emacs terminology yank) one uses the key-command \texttt{C-k}, to delete a whole sentence one uses \texttt{M-k}.\\

Note that whenever one deletes anything bigger than a single character, it is really ``yanked'' to a temporary copy-buffer. One can paste from this buffer with \texttt{C-y}, or move through the buffer with \texttt{M-y}.

\subsubsection{Pasting Text}
After a piece of text has been killed, it goes to a place called the \texttt{kill-ring} which is analogous to the ``clipboard'': you can \emph{yank} an item to restore it from the kill ring with \texttt{C-y}. Unlike the clipboard, however, the kill ring is capable of holding many different items. If the item you want to yank is not placed when you type \texttt{C-y}, type \texttt{M-y} (repeatedly, if necessary) to cycle through previously killed items.

\subsubsection{Undo}
GNU Emacs' undo facility works slightly differently from that of other editors. In most editors, if you undo some changes, then make some new changes, the states formerly accessible with ``redo'' can no longer be recovered! So when using ``undo'' and ``redo'' extensively, one has to be very careful to avoid accidentally clobbering the redo list.\\

GNU Emacs uses a different undo model which does not have this deficiency. After any consecutive sequence of undos, GNU Emacs makes all your previous actions undoable, including the undos. (This will happen whenever a sequence of undos is broken by any other command.)\\

If this sounds complicated, just remember that ``undo'' is always capable of getting you back to any previous state your buffer was in (unless GNU Emacs has run out of memory to store the undo history). The principle here is that GNU Emacs makes it very difficult to accidentally lose your work.

\subsection{Search and Replace Text}
Typing \texttt{C-s} followed by some text starts \emph{incremental search}. GNU Emacs jumps to the next occurrence of whatever the user has typed, as they are typing it (this is similar behavior to what is found in Mozilla Firefox or other web browsers), and all matches visible on the user's screen are highlighted. Within incremental search, the user can type \texttt{C-s} again at any time to jump to the next occurrence.\\

When the user has found what they are looking for, they can either type \texttt{RET} (or use almost any movement command) to exit search at the occurrence found, or \texttt{C-g} (``cancel'') to return to where the search started. If the user exits a search at the found occurrence, they can easily jump back to where the search started with \texttt{C-x C-x} since incremental search sets mark appropriately.\\

These commands help the user issue previously issued queries:\\

\begin{multicols}
  \begin{tabular}{@{}ll@{}}
    \texttt{C-s C-s} & search for most recently searched item.\\
    \texttt{C-s M-p} & previous item in search history.\\
    \texttt{C-s M-n} & next item in search history.\\
    \texttt{C-h k C-s} & guide to more commands available in incremental search.\\
  \end{tabular}
\end{multicols}\\\\

The user can perform a backward incremental search with \texttt{C-r} (all the above commands can be activated similarly from within backward search). At any time during a forward (or backward) search, the user can type \texttt{C-r} (\texttt{C-s}) to switch to a backward (forward) search.

\subsection{Major Modes}
GNU Emacs modes are different behaviors and features which you can turn on or off (or customize, of course) for use in different circumstances. Modes are what make GNU Emacs equally useful for writing documentation, programming in a variety of languages (C, C++, Perl, Haskell, and many more), creating a home page, sending e-mail, reading usenet news, keeping track of your appointments, and even playing games.\\

GNU Emacs modes are simply libraries of Emacs Lisp code that extend, modify, or enhance GNU Emacs in some way.\\

Selecting a major mode changes the meanings of a few keys to become more specifically adapted to the language being edited. The ones that are changed frequently are \texttt{TAB}, \texttt{DEL}, and \texttt{C-j}. The prefix key \texttt{C-c} normally contains mode-specific commands. In addition, the commands which handle comments use the mode to determine how comments are to be delimited. Many major modes redefine the syntactical properties of characters appearing in the buffer.\\

To be able to interpret different types of information, GNU Emacs employs a semi-modal notion similar to the popular \texttt{vim} editor. A major mode defines the context of a buffer. GNU Emacs adapts its behavior to the types of text it edits by entering add-on modes called ``major modes''. Defined major modes exist for ordinary text files, for source code of many programming languages, for HTML documents, for {\TeX} and {\LaTeX} documents, and for many other types of text. Each major mode involves an Emacs Lisp program that extends the editor to behave more conveniently for the particular type of text it covers. Typical major modes will provide some or all of the following common features:

\begin{itemize}
  \item Syntax highlighting (called ``font lock'' in GNU Emacs): using different ``faces'' (font/color combinations) to display keywords, comments, and so forth. Automatic indentation: maintaining consistent formatting within a file.
  \item ``Electric'' features, i.e. the automatic insertion of elements such as spaces, newlines, and parentheses which the structure of the document requires.
  \item Special editing commands: for example, major modes for programming languages usually define commands to jump to the beginning and the end of a function, while major modes for markup languages such as XML provide commands to validate documents or to insert closing tags.
\end{itemize}

The major modes fall into three major groups. The first group contains modes for normal text, either plain or with mark-up. It includes \texttt{text-mode}, \texttt{html-mode}, \texttt{sgml-mode}, \texttt{tex-mode} and \texttt{outline-mode}. The second group contains modes for specific programming languages. These include \texttt{lisp-mode} (which has several variants), \texttt{c-mode}, \texttt{fortran-mode}, and others. The remaining major modes are not intended for use on users' files; they are used in buffers created for specific purposes by GNU Emacs, such as \texttt{dired-mode} for buffers made by Dired,  \texttt{mail-mode} for buffers made by \texttt{C-x m} (the command \texttt{compose-mail}), and \texttt{shell-mode} for buffers used for communicating with an inferior shell process.\\

A major mode can assign specific meanings to certain keys, it makes sense that certain major modes can inherit specific behaviour and key-assignments from parent modes. A list of popular major modes are listed:\\

\begin{multicols}
  \begin{tabular}{@{}ll@{}}
    \texttt{fundamental-mode} & default major mode.\\
    \texttt{text-mode} & for writing text documents.\\
    \texttt{c-mode} & for programming in C.\\
    \texttt{c++-mode} & for programming in C++.\\
    \texttt{python-mode} & for programming in Python.\\
    \texttt{haskell-mode} & for programming in Haskell.\\
    \texttt{lisp-mode} & for programming in Lisp.\\
    \texttt{latex-mode} & for writing {\LaTeX} documents.\\
    \texttt{org-mode} & for organising tasks.
  \end{tabular}
\end{multicols}

\subsubsection{Fundamental mode}
The least specialized major mode is called \emph{Fundamental} mode. This mode has no mode-specific redefinitions or variable settings, so that each GNU Emacs command behaves in its most general manner, and each option is in its default state.

\subsubsection{Text mode}
\subsubsection{C mode}
\subsubsection{C++ mode}
\subsubsection{Python mode}
\subsubsection{Haskell mode}
\subsubsection{Lisp mode}
\subsubsection{{\LaTeX} mode}
\subsubsection{Org mode}
Available from \url{http://orgmode.org/} is for keeping notes, maintaining ToDo lists, doing project planning, and authoring with a fast and effective plain-text system.\\

I currently have \texttt{Org-mode version 7.3} installed.

\subsection{Minor Modes}
Programmers can add extra customized features by using ``minor modes''. While a GNU Emacs editing buffer can use only have one major mode activated at a time, multiple minor modes can operate simultaneously. These may affect documents directly. For example, the major mode for the C programming language defines a different minor mode for each of the popular indent styles. Minor modes may also affect the editing environment instead. For example, ``Winner mode'' adds the ability to undo changes to the window configuration.\\

A minor mode can give new mode-specific behaviour. For instance to spell-check a text document one can use the \texttt{flyspell} minor mode. For a text document, this mode provides a dynamic spell-check over a document as it is being typed. While in a programming-related document the spell-check will only be applied over the comments.\\

\begin{multicols}
  \begin{tabular}{@{}ll@{}}
  \texttt{flyspell-mode} & for spell-checking documents.\\
  \texttt{flymake-mode} & for compiler-checking of interpreted or compiled documents.\\
  \texttt{auto-complete-mode} & will auto-complete snippets of text as you are writing.\\
  \texttt{ibuffer} & filters and then groups the list of buffers currently have open.\\
  \texttt{autopair-mode} & for automatically inserting opposite parenthesis.\\
  \texttt{hs-minor-mode} & for code folding.\\
  \texttt{reftex-mode} & for preparing {\LaTeX} references and bibliographies.
  \end{tabular}
\end{multicols}

\subsection{Extensions}
Extensions are \emph{global} \emph{minor modes} which change (or enhance) GNU Emacs' ability as an editor. The most common extensions are listed below:\\

\begin{multicols}
  \begin{tabular}{@{}ll@{}}
    \texttt{ido-mode} & improves user interaction with the minibuffer.\\
    \texttt{smex} & improves user interaction with the minibuffer.\\
    \texttt{iswitchb} & improves user interaction with the buffer list.\\
    \texttt{ibuffer} & ibuffer is a replacement for the default \texttt{buffermenu}.\\
    \texttt{dired} & a file and directory manager for GNU Emacs.
  \end{tabular}
\end{multicols}

\subsubsection{ido}
The extension \texttt{ido} is an \texttt{M-x} enhancement for GNU Emacs which allows a user to interactively do things with buffers and files.

\subsubsection{smex}
The extension \texttt{smex} is an \texttt{M-x} enhancement for GNU Emacs. Built on top of \texttt{ido-mode}, it provides a convenient interface to a user's recently and most frequently used commands, as well as providing a an interactive interface to all the other commands. This extensions limits commands to those relevant to the active major mode, and shows frequently used commands that have no key bindings.

\subsubsection{iswitchb}
The extension \texttt{iswitchb} is an enhancement for GNU Emacs which provides convenient switching between buffers using substrings of their names. When the user is prompted for a buffer name, they can type in just a substring of the name they want to choose. As the user enters the substring, \texttt{iswitchb} mode continuously displays a list of buffers that match the substring typed.

\subsubsection{ibuffer}
The extension \texttt{ibuffer} is an advanced replacement for \texttt{buffermenu}, which lets the user operate on buffers much in the same manner as \texttt{dired}. The most important \texttt{ibuffer} features are highlighting and various alternate layouts.

\subsection{Help and Documentation}
The first version of GNU Emacs included an innovative help library that could display the documentation for every single command, variable, and internal function. Because of this, proponents of GNU Emacs have described the tools as ``self-documenting'' — in that it presents its own documentation, not only of its normal features but also of its current state, to the user. For example, the user can find out about the command bound to a particular keystroke simply by entering \texttt{C-h k} (which runs the interactive command \texttt{describe-key}), followed by the keystroke. Each function includes a documentation string, specifically to be used for showing to the user on request. The practice of giving functions documentation strings subsequently spread to various programming languages such as Lisp and Java.\\

Further, through GNU Emacs's help system, users can be taken to the actual code for each function; whether a built-in library or an installed third-party library. GNU Emacs also has a built-in tutorial. When GNU Emacs starts with no file to edit, it displays instructions for performing simple editing commands and invoking the tutorial.\\

Some useful help commands are listed below:

\begin{tabular}{@{}ll@{}}
  \texttt{C-h f} & \texttt{describe-function} - describes an Emacs Lisp function.\\
  \texttt{C-h v} & \texttt{describe-variable} - describes an Emacs Lisp variable.\\
  \texttt{C-h k} & \texttt{describe-key} - describes a command bound to a key-binding.\\
  \texttt{C-h c} & \texttt{describe-key-briefly} - shows what command a key is bound to.\\
  \texttt{C-h a} & \texttt{apropos-command} - search command names using a regular expression.\\
  \texttt{C-h m} & \texttt{describe-mode} - describes the current buffer's major mode.\\
\end{tabular}

% \footnote{To search both function and command name spaces, call apropos-command with a empty argument, like this: \texttt{C-u C-h a}}

\section{Terminal and External Commands}
A user can access a terminal from within GNU Emacs, or execute external commands on a region (or buffer).

\subsection{gdb}
GNU Emacs integrates with \texttt{gdb} to provide an IDE (interactive debugging environment). To start a \texttt{gdb} process, the user launches \texttt{M-x} and enters \texttt{gdb} at the interactive minibuffer menu.

\subsection{diff}
GNU Emacs can compare two files and highlight their difference. To start an \texttt{ediff} process, the user launches \texttt{M-x} and enters \texttt{ediff} at the interactive minibuffer menu.

\subsection{gnus}
GNU Emacs can read news, mail, and RSS feeds. To start a \texttt{gnus} process, the user launches \texttt{M-x} and enters \texttt{gnus} at the interactive minibuffer menu.

\subsection{eshell}
The global extension \texttt{eshell} is a command shell written entirely in Emacs Lisp. Everything it does, it uses GNU Emacs' facilities to do. This means that \texttt{eshell} is as portable as GNU Emacs itself. It also means that cooperation with Lisp code is natural and seamless. To start an \texttt{eshell} process. the user launches \texttt{M-x} and enters \texttt{eshell} at the interactive minibuffer menu.\\

\texttt{eshell} is capable of invoking almost any Emacs Lisp function loaded in GNU Emacs.

\subsection{dired}
GNU Emacs includes features and abilities required of a file manager; the global extension \texttt{dired} is a visual directory editor, a computer program for editing file system directories. The available commands are generally more modal than most GNU Emacs commands because \texttt{dired} is a specialized major mode on its own. This extension can perform all expected operations; in operation and use it is akin to an orthodox file manager like ``Midnight Commander''. To start a \texttt{dired} process. the user launches \texttt{M-x} and enters \texttt{dired} at the interactive minibuffer menu.\\

\texttt{dired-mode} is the mode of a \texttt{dired} buffer. It shows a directory (folder) listing that you can use to perform various operations on files and subdirectories in the directory. The operations you can perform are numerous, from creating subdirectories to byte-compiling files, searching files, and of course visiting (editing) files.\\

Several Emacs Lisp scripts have been developed to extend \texttt{dired} functionalities in GNU Emacs. In combination with \texttt{tramp} it is able to access remote file systems for editing files by means of \texttt{SSH}, \texttt{FTP}, \texttt{telnet} and many other protocols, as well as the capability of accessing as another user for editing files with restricted permissions (such as administrator access) in the same session. There are also functionalities that make it possible to rename multiple files via search and replace or apply regular expressions for marking (selecting) multiple files.

\subsubsection{The Basics}
To initiate a \texttt{dired} buffer type \texttt{C-x d} and you will be prompted for a directory path starting in the directory of your current open buffer. Navigate the path you want to open and return to open it. If you haven't extended \texttt{insert text} with additional modes, you'll get something that looks like a terminal directory listing (or the command \texttt{ls -lna} for the terminally inclined).

\subsubsection{Navigation}
Movement in \texttt{dired} is exactly like any buffer. For long directories it is useful to navigate with \texttt{isearch} (\texttt{C-s} and \texttt{C-r}). To open a folder press \texttt{f} (which executes the interactive \texttt{find-file} command). To go up a directory type \texttt{\^}. Every time you open a new directory a new buffer is opened in the background for you and always accessible until you kill it.\\

Top view a file in \texttt{read-only} mode, press \texttt{v}. To return to the \texttt{dired} buffer press \texttt{q}. The character \texttt{o} will open the file in another window while RET or \texttt{f} will open the file in the current window. At anytime in a \texttt{dired} buffer the user can sort the list of files (and directories) by typing \texttt{s}.

\subsubsection{Deleting Files}
To delete files, the user must first mark the files for deletion by navigating to the file or folder and pressing \texttt{d} which will place a capital D to the left of the file. When you are done with your selections \texttt{x} will initiate the actions and you will be prompted to confirm your actions. Typing \texttt{#} marks all the autosave files for deletion with a D status and typing, \texttt{\~} marks all the backup files for deletion with the same status. The user can press \texttt{u} to unmark the selected files.

\subsubsection{Copying and Renaming Files}
To copy a file, the user must first navigate to it and press \texttt{C} (capital C). GNU Emacs will ask the user for the name of the file to copy to. Enter will execute the copy. If the user types \texttt{C4} from the file current file, the current file and the next three will be copied. To rename a file type \texttt{R}.\\

In summary, here is a list of the basic commands:

\begin{tabular}{@{}ll@{}}
  \textbf{Key} & \textbf{Function}\\
  \texttt{Enter} & Open selected file.\\
  \texttt{f} & Open selected directory.\\
  \texttt{a} & Open selected file and close \texttt{dired} buffer.\\
  \texttt{d} & Mark selected files for deletion.\\
  \texttt{m} & Mark selected files.\\
  \texttt{u} & Unmark selected file.\\
  \texttt{U} & Unmark \emph{all} marked files.\\
  \texttt{t} & Toggle between marked/unmarked.\\
  \texttt{\%m} & Mark by regular expression.\\
  \texttt{g} & Refresh directory listing.\\
  \texttt{\^} & Go to parent directory.\\
  \texttt{+} & Create new directory.\\
  \texttt{Z} & Compress/Decompress selected file by \texttt{gzip}.\\
  \texttt{x} & Delete selected files.\\
  \texttt{o} & Open selected file in a split window, so the directory is still visible.\\
  \texttt{q} & Close current directory.\\
  \texttt{C} & Copy selected file.\\
  \texttt{R} & Rename selected file.\\
  \texttt{D} & Delete selected file.
\end{tabular}

\subsubsection{File Compression}
The user can compress and uncompress files from \texttt{dired} by typing \texttt{Z} (which runs the command \texttt{dired-do-compress}). This command is universal in that it will uncompress a compressed file or compress it if it's not compressed already.

\subsubsection{Diffs}
A useful feature of \texttt{dired} is that it allows the user to compare a file with its respective backup file. The user need just find the file to compare and press \texttt{=}. The user will be presented a buffer showing you the difference in the two files.

\section{Customising GNU Emacs with Emacs Lisp}
The \emph{real} advantage and power comes not from the default (vanilla) GNU Emacs, but from extending the editor through its use of Emacs Lisp.  So far we have introduced the simple editing features of emacs which are directly accessible without requiring much user input.  Not only is GNU Emacs an editor, it is also a file manager through \texttt{M-x dired}.  In dired, you are "editing" a list of the files in a directory and (optionally) its subdirectories, in the format of `ls -lR'. You can also configure GNU Emacs to receive and send email for you. Or configure GNU Emacs to be your IRC client.\\

GNU Emacs Lisp is largely inspired by Maclisp, and a little by Common Lisp. If you know Common Lisp, you will notice many similarities. However, many features of Common Lisp have been omitted or simplified in order to reduce the memory requirements of GNU Emacs. Sometimes the simplifications are so drastic that a Common Lisp user might be very confused. We will occasionally point out how GNU Emacs Lisp differs from Common Lisp. If you don't know Common Lisp, don't worry about it; this manual is self-contained.\\

A certain amount of Common Lisp emulation is available via the \texttt{cl} library.

\subsection{Notation}
In Lisp, the symbol \texttt{nil} has three separate meanings:
\begin{enumerate}
  \item it is a symbol with the name `\texttt{nil}';
  \item it is the logical truth-value false; and
  \item it is the empty list - the list of zero elements. When used as a variable, \texttt{nil} always has the value \texttt{nil}.
\end{enumerate}

As far as the Lisp reader is concerned, `\texttt{()}' and `\texttt{nil}' are identical: they stand for the same object, the symbol \texttt{nil}. The different ways of writing the symbol are intended entirely for human readers. After the Lisp evaluator has read either `\texttt{()}' or `\texttt{nil}, there is no way to determine which representation was actually written by the programmer.\\

In this manual, we write \texttt{()} when we wish to emphasize that it means the empty list, and we write \texttt{nil} when we wish to emphasize that it means the truth-value false. That is a good convention to use in Lisp programs also.

\begin{lstlisting}
  (cons 'foo ()) ;; emphasize the empty list
  (setq foo-flag nil) ;; emphasize the truth-value false
\end{lstlisting}   
     
In contexts where a truth value is expected, any non-\texttt{nil} value is considered to be true. However, \texttt{t} is the preferred way to represent the truth value true. When you need to choose a value which represents true, and there is no other basis for choosing, use \texttt{t}. The symbol \texttt{t} \emph{always} has the value \texttt{t}.\\

In Emacs Lisp, \texttt{nil} and \texttt{t} are special symbols that always evaluate to themselves. This is so that you do not need to quote them to use them as constants in a program. An attempt to change their values results in a setting-constant error.

\subsection{Functions}

\subsection{Variables}

\subsection{Macros}

\end{document}
